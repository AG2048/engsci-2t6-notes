\documentclass{article}

\usepackage{amsmath, amsthm, amssymb, amsfonts}
\usepackage{thmtools}
\usepackage{graphicx}
\usepackage{setspace}
\usepackage{geometry}
\usepackage{float}
\usepackage{hyperref}
\usepackage[utf8]{inputenc}
\usepackage[english]{babel}
\usepackage{framed}
\usepackage[dvipsnames]{xcolor}
\usepackage{tcolorbox}
\usepackage{bm}

% package for mathds
\usepackage{dsfont}

\colorlet{LightGray}{White!90!Periwinkle}
\colorlet{LightOrange}{Orange!15}
\colorlet{LightGreen}{Green!15}

\newcommand{\HRule}[1]{\rule{\linewidth}{#1}}

\declaretheoremstyle[name=Theorem,]{thmsty}
\declaretheorem[style=thmsty,numberwithin=section]{theorem}
\tcolorboxenvironment{theorem}{colback=LightGray}

\declaretheoremstyle[name=Proposition,]{prosty}
\declaretheorem[style=prosty,numberlike=theorem]{proposition}
\tcolorboxenvironment{proposition}{colback=LightOrange}

\declaretheoremstyle[name=Principle,]{prcpsty}
\declaretheorem[style=prcpsty,numberlike=theorem]{principle}
\tcolorboxenvironment{principle}{colback=LightGreen}

\setstretch{1.2}
\geometry{
    textheight=9in,
    textwidth=5.5in,
    top=1in,
    headheight=12pt,
    headsep=25pt,
    footskip=30pt
}

% ------------------------------------------------------------------------------

\begin{document}

% ------------------------------------------------------------------------------
% Cover Page and ToC
% ------------------------------------------------------------------------------

\title{ \normalsize \textsc{}
		\\ [2.0cm]
		\HRule{1.5pt} \\
		\LARGE \textbf{\uppercase{Lab 3}
		\HRule{2.0pt} \\ [0.6cm] \LARGE{ECE356} \vspace*{10\baselineskip}}
		}
\date{}
\author{\textbf{Andy Gong} \\ 
		1008848482}

\maketitle
\newpage

\tableofcontents
\newpage

% ------------------------------------------------------------------------------
\section{Preparation} % The big UNIT STEP FUNCTION's symbol is the big ONE(t): $\mathcal{1}(t)$
Assuming $\theta = 0$ and hence $D(s) = 0$, find final value of $v$ if input is $v_m(t) = V_0 \cdot \mathds{1}(t)$, and $v(0) = 0$.

$$V(s) = V_m(s) \cdot  \frac{a}{s + b}$$

$$V(s) = V_0 \cdot \frac{a}{s \cdot (s + b)}$$

Using FVT:

$$\lim_{t \to \infty} v(t) = \lim_{s \to 0} s \cdot V(s) = \lim_{s \to 0} s \cdot V_0 \cdot \frac{a}{s \cdot (s + b)} = V_0 \cdot \frac{a}{b}$$

\end{document}