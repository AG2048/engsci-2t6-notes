\documentclass{article}

\usepackage{amsmath, amsthm, amssymb, amsfonts}
\usepackage{thmtools}
\usepackage{graphicx}
\usepackage{setspace}
\usepackage{geometry}
\usepackage{float}
\usepackage{hyperref}
\usepackage[utf8]{inputenc}
\usepackage[english]{babel}
\usepackage{framed}
\usepackage[dvipsnames]{xcolor}
\usepackage{tcolorbox}
\usepackage{bm}

% package for mathds
\usepackage{dsfont}

\colorlet{LightGray}{White!90!Periwinkle}
\colorlet{LightOrange}{Orange!15}
\colorlet{LightGreen}{Green!15}

\newcommand{\HRule}[1]{\rule{\linewidth}{#1}}

\declaretheoremstyle[name=Theorem,]{thmsty}
\declaretheorem[style=thmsty,numberwithin=section]{theorem}
\tcolorboxenvironment{theorem}{colback=LightGray}

\declaretheoremstyle[name=Proposition,]{prosty}
\declaretheorem[style=prosty,numberlike=theorem]{proposition}
\tcolorboxenvironment{proposition}{colback=LightOrange}

\declaretheoremstyle[name=Principle,]{prcpsty}
\declaretheorem[style=prcpsty,numberlike=theorem]{principle}
\tcolorboxenvironment{principle}{colback=LightGreen}

\setstretch{1.2}
\geometry{
    textheight=9in,
    textwidth=5.5in,
    top=1in,
    headheight=12pt,
    headsep=25pt,
    footskip=30pt
}

% ------------------------------------------------------------------------------

\begin{document}

% ------------------------------------------------------------------------------
% Cover Page and ToC
% ------------------------------------------------------------------------------

\title{ \normalsize \textsc{}
		\\ [2.0cm]
		\HRule{1.5pt} \\
		\LARGE \textbf{\uppercase{Lab 4}
		\HRule{2.0pt} \\ [0.6cm] \LARGE{ECE356} \vspace*{10\baselineskip}}
		}
\date{}
\author{\textbf{Andy Gong} \\ 
		1008848482}

\maketitle
\newpage

\tableofcontents
\newpage

% ------------------------------------------------------------------------------
\section{Preparation} % The big UNIT STEP FUNCTION's symbol is the big ONE(t): $\mathcal{1}(t)$
The plant has no poles at zero, so we need to have a pole at zero for the controller. 

Write the transfer function from $R(s)$ to $E(s)$, assuming $D(s) = 0$. And the transfer function from $D(s)$ to $E(s)$, assuming $R(s) = 0$. 

For the first case:
$$E(s) = R(s) - C(s)G(s)E(s)$$
$$E(s) = \frac{R(s)}{1 + C(s)G(s)}$$

For the second case:
$$E(s) = 0 - \left(C(s)E(s) + D(s)\right) G(s)$$
$$E(s) = -\frac{D(s)G(s)}{1+C(s)G(s)}$$

Subbing in $G(s) = \frac{a}{s+b}$ and $C(s) = K(1 + \frac{1}{T_I s})$:
$$E(s) = \frac{R(s)}{1 + \frac{a}{s+b} \cdot K(1 + \frac{1}{T_I s})}$$
$$E(s) = \frac{R(s) (s+b) (T_I s)}{(s+b) (T_I s) + a K (T_I s + 1)}$$

$$E(s) = \frac{-D(s) \frac{a}{s+b}}{1 + \frac{a}{s+b} \cdot K(1 + \frac{1}{T_I s})}$$
$$E(s) = \frac{-D(s) a (T_I s)}{(s+b) (T_I s) + a K (T_I s + 1)}$$

Setting $R(s) = \frac{\overline{v}}{s}$ and $D(s) = \frac{\overline{d}}{s}$, we get:
$$E(s) = \frac{\overline{v} (s+b) T_I}{(s+b) (T_I s) + a K (T_I s + 1)}$$
$$E(s) = \frac{-\overline{d} a T_I}{(s+b) (T_I s) + a K (T_I s + 1)}$$

If the limit of $\lim_{t \to \infty} e(t)$ exists, then we can use the final value theorem:
$$\lim_{t \to \infty} e(t) = \lim_{s \to 0} s E(s)$$
$$\lim_{t \to \infty} e(t) = \lim_{s \to 0} s \cdot \frac{\overline{v} (s+b) T_I}{(s+b) (T_I s) + a K (T_I s + 1)} = 0$$
$$\lim_{t \to \infty} e(t) = \lim_{s \to 0} s \cdot \frac{-\overline{d} a T_I}{(s+b) (T_I s) + a K (T_I s + 1)} = 0$$

The controller asymptotically tracks the reference even if the disturbance is present, as error converges to zero.

\end{document}