\section{L04 Combos and stuff}
Sep 20, 2023
\begin{itemize}
	\item Combos
		\begin{theorem}
			Assume event $A$ can happen $m$ ways and $B$ $n$ ways:\\
			Rule of Product: A and B can happen $m\times n$ ways\\
			Rule of Sum: A or B can happen $m + n$ ways
		\end{theorem}
		\begin{example}
			You have 2 English, 5 Greek, and 10 French books.\\
			Number of ways to select 2 different bats: $2\times 5 + 2\times 10 + 5\times 10$ \\
			Number of ways to select any 2 bats: $17\times 16$
		\end{example}
		\begin{theorem}
			Permutations:\\
			$P(n,r)$ Select $r$ objects out of $n$ distinct objects where order matters\\
			$P(n,r) = n \left( n-1 \right) \left( n-2 \right) \ldots \left( n-r+1 \right) = \frac{n!}{\left( n-r \right) !}$
			Picking Green THEN Blue is different from picking Blue THEN Green.
		\end{theorem}
		\begin{example}
			How many ways $n$ people can sit on a round table?\\
			If it's a straight table, it would be $P\left( n,n \right) = n!$ \\
			However, since it's a round table, rotation makes a lot of "different" positions actually the same. So we FIX one person, and it becomes:
			$\frac{n!}{n} = \left( n-1 \right) !$
		\end{example}
		\begin{theorem}
			If not all objects are distinct, but:\\
			\begin{align*}
				a_1 \text{ first}\\
				a_2\ldots\\
				a_3\ldots\\
				ak\ldots\\
			.\end{align*}
			Then the combinations become:
			\begin{equation}
				\frac{n!}{a_1! \cdot a_2! \cdot a_3! \cdot \ldots \cdot  a_{k}!}
			\end{equation}
			The reason is now we reduce all possible combinations where only $a_i$ objects shuffle around, which doesn't create new combinations\ldots
		\end{theorem}
	\item Combinatorial argument
		\begin{example}
			Show \[
			\frac{\left( k! \right) !}{\left( k! \right) ^{\left( k-1 \right) !}}
			.\] 
			Is an integer by a combinatorial argument.\\

			Lets assume there are: 
			\begin{align*}
				k~ &\text{items of 1st kind}\\
				k~ & \text{2nd}\\
				k~ & \ldots\\
				k~ & \left( k-1 \right) ! \text{ kind}
			.\end{align*}
			I have $k!$ total items. And we want to permute them. \\
			Lets call $k! = n$, ways to permute would be:
			\[
			\frac{n!}{k! \cdot k! \cdot k! \cdot \ldots \cdot k!} = \frac{n!}{\left( (k!) \right) ^{\left( k-1 \right) !}} = \frac{\left( k! \right) !}{\left( k! \right) ^{\left( k-1 \right) !}}
			.\] 
		\end{example}
		\begin{theorem}
			Ways to permute $r$ objects out of $n$ with unlimited repetition = $n^{r}$
		\end{theorem}
		\begin{example}
			Among 10 billion numbers $[1, \ldots, 10,000,000,000]$, how many contain the digit "1"?\\
			We can convert numbers into a list of 10 numbers, each picking from 0 to 9. This represents all numbers from 0 to 9,999,999,999\\

			How many DO NOT contain "1"? That's just $9^{10}$. We are picking from 0, 2, 3, 4, 5, 6, 7, 8, 9 for 10 positions.

			We then subtract 10 billion with $9^{10}-1$. We must remove the "non-1" number of "0" cuz it's not in the original range.
		\end{example}
		\begin{theorem}
			Combinations: select $r$ objects out of $n$ distinct objects but order does not matter\\
			 \begin{equation}
			 	C\left( n,r \right) = \frac{P\left( n,r \right) }{r!} =  \frac{n!}{r! \cdot  \left( n-r \right) !}
			 \end{equation}
			 We call this "n choose r" = binomial coefficient\\
			 Side note: 
			 \begin{equation}
				 \left( x+y \right) ^{r} = \sum_{i=0}^{r} \begin{pmatrix} r \\ i \end{pmatrix} x^{i}y^{r-i}
			 \end{equation}

		\end{theorem}
		\begin{example}
			How many diagonals does a 10-sided polygon have?\\
			$C(10,2) - 10$, we choose any 2 corners out of 10 possible corners. Subtract the 10 original sides.
		\end{example}
		\begin{example}
			In how many ways we can select 3 distinct numbers fron $[1\ldots 300]$ such that their unique sum is divisble by 3?\\
			100 numbers / 3 leaves remainder 0, add them together no remainder\\
			100 numbers leave remainder 1 add together no remainder\\
			100 numbers leave remainder 2 add together no remainder\\
			Then we need to add combinations of picking 1 number from each\\
			Answer:\\
			$C\left( 100, 3 \right)  + C\left( 100,3 \right) + C\left( 100,3 \right)  + 100^3$
		\end{example}
		\begin{example}
			11 scientist work on a secret project that they lock in a cabinet such that at least 6 scientists need to be present to unlock it.\\
			What is the minimum number of cabinet locks to make this work?\\
			What is the minimum number of keys per scientist?\\
			\begin{enumerate}
				\item have a lock for every possible combination of FIVE scientists. So 5 scientists can't open it. $C\left( 11, 5 \right) $. $\forall \text{ group of 5}, \exists \text{ unique lock they cannot unlock}$
				\item $C\left( 10, 5 \right) $ Keys are required. For any scientist, they should be able to join any group of 5 from the remaining 10 people, and have a key that fits the job.
			\end{enumerate}
		\end{example}
\end{itemize}

\section{L04 Master Method}

\begin{itemize}
	\item Recurrence
	\begin{theorem}
		Let $a \ge 1,~ b > 1$ and $f(n)$ be recurrent function:\\
		\begin{equation}
			T(n) = a T\left( \frac{n}{b} \right) + f(n)
		\end{equation}
		Has solution:\\
		if $f(n)=\mathcal{O}\left( n^{\log_b a - \epsilon} \right) $ for $\epsilon > 0$ then $T(n) = \Theta \left( n^{\log_b a} \right) $ \\
		if $f(n) = \Theta\left( n^{\log_b a} \right) $ Then $T(n) = \Theta\left( n^{ \log_b a} \log n \right) $ \\
		if $f(n)=\Omega\left( n^{\log_b a + \epsilon} \right) $ for $\epsilon > 0$ and $af\left( \frac{n}{b} \right)  \le  cf\left( n \right) $ for $0 < c < 1$ then $T(n) = \Theta\left( f\left( n \right)  \right) $\\
	\end{theorem}
	\begin{example}
		MERGESORT (A, p, v)\\
		if $P < r$:\\
		    $q = \left\lfloor \frac{p + r}{2} \right\rfloor$\\
			Mergesort(A, q+1, r) (This is O(n))\\
			Mergesort(A, P, q) (O(n))\\
			MERGE(A, p, q, r) (O(n))\\
		Time complexity would just be:
		\begin{equation}
			T(n) = 2T\left( \frac{n}{2} \right)  + \mathcal O(n) = \Theta\left( n \log n \right) 
		\end{equation}
	\end{example}
	\begin{example}
		If function is upper bounded
		If function is tightly bounded by something, function itself matters.\\
		If function is lower bounded, then time complexity is just f(n)
	\end{example}
	\begin{example}
		$T\left( n \right) = 2T\left( \frac{n}{2} \right) + \mathcal O(n)$ \\
		This is 2nd case, where:
		\begin{align*}
			a &=  2 \\
			b &= 2 \\
			f\left( n \right) &= \mathcal O \left( n \right) = \Theta\left( n^{\log_2 2} \right)  \\
		.\end{align*}
		So $T(n) = \Theta\left( n \log n \right) $
	\end{example}
	\begin{example}
		$T(n) = T\left( \frac{2n}{3} \right) + 1$\\
		This is case 2, where
		\begin{align*}
			a &=  1 \\ b &= \frac{3}{2} \\ f\left( n \right) &= 1 \\
			\log_{\frac{3}{2}}1 &= 0 \\
			f\left( n \right)  &= \Theta\left( 1 \right)  \\
		.\end{align*}
		From this, $T(n) =  \Theta\left( n^{0} \log n \right) = \Theta\left( \log n \right) $
	\end{example}
	\begin{example}
		$T(n) = 9T\left( \frac{n}{3} \right) + n$ 
		\begin{align*}
			a &=  9 \\
			b &=  3 \\
			f(n) &=  n \\
			\log_3 9 &=  2 \\
			f(n) &=  \mathcal O\left( n^{2-\epsilon} \right)  \\
		.\end{align*}
		From this, $T(n) =  \Theta\left( n^2 \right)$
	\end{example}
	\begin{example}
		$T(n) = 3T\left( \frac{n}{4} \right) + n \log n$ 
		\begin{align*}
			a &=  3 \\
			b &= 4 \\
			f(n) &=  n \log n \\
			\log_4 3 &= 0.793\\
			f\left( n \right)  &=  \Omega\left( n^{\log_4 3 + 0.2} \right)  \\
		.\end{align*}
		From this:\\
		$T(n) = \Theta\left( n \log n \right)$ IF $3f\left( \frac{n}{4} \right) = \frac{3}{4} n \log \frac{n}{4} \le  \frac{3}{4} n \log n$. Here we used $c \ge \frac{3}{4}$ 

	\end{example}
\item SUBSTITUTION
	\begin{theorem}
		SUBSTITUTION, basically induction\\
		We just guess the answer and prove by induction
	\end{theorem}
	\begin{example}
		\begin{equation}
			T(n) = 2 T\left( \left\lfloor \frac{n}{2} \right\rfloor \right) + n
		\end{equation}
		For this, we just guess $\mathcal O \left( n \log n \right) $ \\
		HYPOTHESIS: works for $< n$\\
		STEP: it works for $\frac{n}{2}$: $T\left( \left\lfloor \frac{n}{2} \right\rfloor \right) \le c \frac{n}{2} \log \frac{n}{2}$\\
		\begin{align*}
			T\left( n \right)  & \le 2c \frac{n}{2} \log \frac{n}{2} + n\\
			&= cn\log n - cn \log 2 + n \\
			&= cn \log n - cn + n \\
			&= cn \log n + \left( 1-c \right) n \\
			& \le cn \log n\\
			\forall c \ge 1
		.\end{align*}
		BASIS: $T(1) = 1 \le  cn \log n = c 1 \log 1 = 0$ doesn't work\\
		BASIS: $T(2) = 4 \le  c 2 \log 2 = 2c$ \\
		BASIS: $T(3) = 5 \le  c 3 \log 3 = c 3 \cdot  0.477$\\
		So we pick $c > 3, \forall n > 3$
	\end{example}
\item RECURSION TREE:
	\begin{example}
		$t(n) = T\left( \frac{n}{4} \right) + T\left( \frac{2n}{3} \right) + n$ \\
		We can't use other methods for this. Master method doesn't work, and induction is iffy.\\
		We draw a tree:
		\begin{align*}
			n\\
			\frac{n}{4} ~\frac{2n}{3}\\
\frac{n}{16}~ \frac{2n}{12}~\frac{2n}{12}~\frac{4n}{9}\\
\ldots
		.\end{align*}
		This is how the algorithm will be going. At some point the algorithm will hit a number and STOP. The rightmost path will be the LONGEST path, since the rightmost path we go on $\frac{2}{3}$.\\
		At every level the algorithm splits, $\mathcal O\left( n \right) $ is introduced. In every layer of the tree, there is $\mathcal O(n)$ work.\\
		To use this tree to help us understand how much work is required:
		\begin{equation}
			T(n) = h * \mathcal O(n)
		\end{equation}
		The work this does is the HEIGHT OF THE TREE multiplied by $\mathcal O(n)$\\
		Question now: how can we find the height of the tree?\\
		Since we keep going  $\frac{2}{3}$ downward forever on the longest path, it would be (since eventually we hit an element):  \[
			\left( \frac{2}{3} \right) ^h \cdot  n = 1
		.\] 
		Meaning that $h = \log_{\frac{3}{2}}n = \mathcal O(\log n)$\\
		So the entire function would be $\mathcal O\left( n \log n \right) $
	\end{example}
\end{itemize}
