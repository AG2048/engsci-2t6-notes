\section{L05 Graphs}
Sep 26, 2023
\begin{itemize}
	\item Graphs
		\begin{definition}
			Graphs: $G(V, E)$ means a graph of Vectors ane Edges\\
			There exists:
			 \begin{enumerate}
				 \item Directed and Undirected
				 \item Weighted Edges (Distance, penalty, profit)
				 \item Connected Graph: $\forall u, v \in V,~ \exists u \to v$ (forall nodes u, v, in the graph, there exists a path from u to v)
				 \item Strongly connected component: a subgraph that is connected.
				 \item Bipartide Graph: There are two sets of vertices, and the two sets are interconnected.
				 \item Clique (completed graph): Every possible pairs of vertices are connected. $\exists \left( u, v \right) \in E~\forall u,v \in V$. Also, number of edges would be $\frac{V(V-1)}{2}$
				 \item Directed Acyclic Graph (DAG): A directed graph with no cycles. Starting from one vertex you cannot reach back to the same vertex.
				 \item Vertex Degree: Number of edges connected to a vertex. (There is a difference between "in-degree" and "out-degree" in directed graphs) (We can prove that sum of all vertex degrees = twice the number of edges. We can prove by induction, but this should be quite obvious)
			\end{enumerate}
		\end{definition}
		\begin{definition}
			Adjacency List:\\
			We create a list of size $\lVert V \rVert $ and each element in the list is a list of vertices that are connected to the vertex.\\
		\end{definition}
		\begin{definition}
			Adjacency Matrix:\\
			We create a matrix of size $\lVert V \rVert \times \lVert V \rVert$ and each element in the matrix is a boolean value that indicates whether the two vertices are connected.\\
			We can also store the weight of the edge in the matrix.\\
			For undirected graphs, the matrix is symmetric.\\
			But if the graph is directed, then it might not be symmetric.\\
			For a directed graph, how to find the inverse graph (Inverse means directed connections are all flipped)? JUST TRANSPOSE THE MATRIX.\\
		\end{definition}
		\begin{example}
			For each type of representation:
			$$
			\begin{bmatrix} & TIME & MEM\\AL & \mathcal O(V) & \mathcal O(V+E) \\ AM & \mathcal O(1) & \mathcal O(V^2) \end{bmatrix} 
			$$
			Therefore, Adj Matrix is not really good for sparce Graphs (where not enough connections)
		\end{example}
	\item Trees
		\begin{definition}
			What is a tree: A undirected, connected acyclic graph.\\
			Key is no cycle. Connected. No direction.\\
			Whenever we draw trees, we want to show some sort of hierarchy, to make things clearer.\\
			ROOT: the top node of the tree.\\
			CHILD: a node that is connected to a parent.\\
			PARENT: a node that is connected to a child.\\
			SIBLING: nodes that have the same parent.\\
			SUBTREE: a tree that is connected to a node.\\
			DEPTH: the number of edges from the root to the node.\\
			HEIGHT: the number of edges from the node to the deepest leaf.\\
			K-AVARY TREE: a tree where each node has at most k children. One example is a binary tree, which is a 2-avary tree.\\
			FULL TREE: a tree where each node has exactly k children. a full binary tree is a binary tree where each node has exactly 2 children.\\

		\end{definition}
		\begin{theorem}
			All statements are equivalent:
			\begin{enumerate}
				\item $G$ is a tree\\
				\item $\forall$ two vectors $u, v \in V$, there is exactly one path from $u$ to $v$\\
				\item $G$ is connected and $\lVert E \rVert = \lVert V \rVert - 1$\\
				\item $G$ is acyclic and $\lVert E \rVert = \lVert V \rVert - 1$\\
				\item $G$ is connected but if any edge removed, it becomes disconnected.\\
				\item $G$ is cryclic but if any edge added, it creates a cycle.
			\end{enumerate}
		\end{theorem}
\end{itemize}
