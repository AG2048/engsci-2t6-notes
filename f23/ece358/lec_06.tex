\section{L06 QuickSort}
Sep 27, 2023
\begin{itemize}
	\item Quick Sort (Divide and Conquer)
		\begin{lstlisting}
QS(in, left, right)
	pivot = partition(in, left, right)
	if pivot > left
		QS(in, left, pivot)
	if pivot < right
		QS(in, pivot+1, right)
int partition(in, left, right)
	es = lelft
	pivot = in[left]
	for i = left+1 to right
		if (in[i] <= pivot)
			es = es + 1
			swap(in[i], in[es])
	swap(in[left], in[es])
	return es
		\end{lstlisting}
	\begin{example}
		Take an array. Break the array into 2 parts. All values bigger than P, less equal to P. Then, recursively sort the 2 parts.\\ Also, the partition takes $\mathcal O(n)$\\
		Take the array of unsorted numbers, we pick the leftmost as pivot. Then we move thru the whole list.\\
		A snapshot of the array is shown below.\\
		LEFT - pivot - i - RIGHT\\
		Example:\\
		26, 23, 35, 29, 19, 12, 22\\
		ls = 26\\
		go thru each by each, round 19 < ls\\
		swap and result is:\\
		26 19 35 29 33 12 22\\
		ls increment by 1 to pos of 35\\
		Then:
		26 19 12 22 33 35 29\\
		This finishes the partition.\\
		The partition is now around 29\\
		Left list:\\
		26 19 12 22\\
		22 12 19 26\\
		Split LEFT LEFT:\\
		19, 12, 22\\
		LEFT LEFT LEFT:\\
		12, 19\\
		LEFT LEFT LEFT LEFT:\\
		12\\
		LEFT LEFT LEFT RIGHT:\\
		19\\
		LEFT LEFT RIGHT:\\
		22\\
		LEFT RIGHT:\\
		26\\
		Right list:\\
		35 29\\
		29 35\\
		29\\
		35\\
	\end{example}
	\begin{theorem}
		Quick Sort: $T(n) = T(n-1) + \Theta(n)$ because we split tree, each split we take  $\Theta (n)$ times, and recursion.\\
		This ends to be  $\mathcal O(n^2)$\\
		Worse case happens when we split $\mathcal O(n)$ list into 2 branches of  $\mathcal O(1), \mathcal O(n-1)$
	\end{theorem}
	\begin{theorem}
		Best Case: When it's split exactly in the middle.\\
		We split the tree into $\frac{n}{2}$ and $\frac{n}{2}$. When the partition is always the median\\
		This case: $T(n) = 2T\left( \frac{n}{2} \right)  + \mathcal O(n) = \Theta(n \log n)$
	\end{theorem}
	\begin{theorem}
		Average Case: we split randomly, not necessairly half and half.\\
		Assume we split  $\frac{n}{8}$ and $\frac{7n}{8}$ \\
		$T(n) = T\left( \frac{n}{8} \right)  + T\left( \frac{7n}{8} \right) + \mathcal O(n)$\\
		Then when we split the tree, the longest branch will have $\frac{7n}{8} \times \frac{7n}{8} \times \frac{7n}{8} \times \ldots$. And we know that each layer of the tree will require $\mathcal O(n)$ to run.\\
		Now we need to find the HEIGHT of the tree, where  $\frac{7}{8} ^{h} \times n = 1$, then $h = \log_{\frac{8}{7}} n$, and $h = \mathcal O(\log n)$
		Then $T(n) = \mathcal O(n \log n)$ since $T(n) = h \cdot  \mathcal O(n)$\\
		NOTE: EVEN IF WE MAKE A 79999:80000 SPLIT, ASYMPTOTICALLY IT IS STILL $\mathcal O(n \log n)$
	\end{theorem}
\item We can do randomized partition to ensure that we get the average case.\\
	\begin{lstlisting}
rand-partition(in, left, right)
	l = random(left, right)
	swap (in[l], in[left])
	return partition
	\end{lstlisting}
\end{itemize}
\section{L06 QuickSort-Continued}
Sep 27, 2023
\begin{itemize}
	\item Worse Case Analysis:
		\begin{theorem}
			$T(n) = \max_{1 \le  q \le  n-1} \left\{ T\left( q \right) + T\left( n-q \right) + \Theta(n)\right\}= \mathcal O(n^2)$\\
			Step: $T(n) \le  \max_{1 \le  q \le  n-1} c \left\{ q^2 + \left( n-q \right) ^2 \right\} + \Theta(n)$\\
			As 2nd derivative of q is positive max for q = 1 or 1 = n-1, let q = 1\\
			$$
			\max \left\{ q^2 + (n-q)^2 \right\} \le  1^2 + (n-1)^2 = n^2 - 2(n-1)
			$$
			Therefore $T(n) \le  c(n^2 - 2(n-1)) + \Theta(n) = \mathcal O(n^2)$
		\end{theorem}
		\begin{example}
			Expected case of random QS\\
			$$
T(n) = \frac{1}{n} \left( T(1) + T(n-1)+  \sum_{q=1}^{n-1} T(q) + T(n-q) \right) + \Theta(n)
			$$
			But
			$$
			\frac{1}{n} \left( T(1) + T(n-1) \right)  \le  \frac{1}{n} \left( \Theta(1) + \Theta(n^2) \right) = \mathcal O(n)
			$$
			So we bring in the largest term\\
			$$
T(n) = \frac{1}{n} \sum_{q}^{n-1} T(q) + T(n-q) + \Theta(n) = \frac{2}{n} \sum_{q=1}^{n-1} T(q) + \Theta(n)
			$$
			Use substitution to show that $T(n) = a n \log n + b$ for some $a,b$ \\
			STEP\\
			$$
T(n) \le  \frac{2}{n} \sum_{k=1}^{n-1} a k \log k + b + \Theta(n) = \frac{2a}{n} \sum_{k=1}^{n-1} k \log k + \frac{2b}{n}(n-1) +  \Theta(n) \le  a n \log n - \frac{a}{4}n + 2b + \Theta(n) = a n \log n + b + \left( \Theta(n) + b - \frac{a}{4} n  \right) < an \log n + b
			$$ Note last step requires a large enough $a$ value.
		\end{example}
		\begin{lemma}
			$$
\sum_{k=1}^{n-1} k \log k \le  \frac{1}{2} n^2 \log n - \frac{1}{8} n^2
			$$
			Sometimes theorem can be hard to proof. So we just assume something is true
			\[
			\sum_{k=1}^{\left\lfloor \frac{n}{a} \right\rfloor-1} k \log k + \sum_{k=\left\lceil \frac{n}{2} \right\rceil }^{n-1} k \log k \le  \log \frac{n}{2} \sum_{1}^{\left\lfloor \frac{n}{2} \right\rfloor} k + \log n \sum_{k = \left\lfloor \frac{n}{q} \right\rfloor}^{n-1} k = \ldots
			.\] 
		\end{lemma}
\end{itemize}
