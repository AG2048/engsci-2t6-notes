\newpage
\section{Directional Derivatives and Gradient Functions}
March 24, 2023
\begin{itemize}
	\item A partial derivative:
		\begin{definition}
			A partial derivative of $f$ on $x$, denoted as $f_x = \frac{\partial f}{\partial x} $ is defined as:
			\begin{equation}
				\lim_{h \to 0} \frac{f\left( x+h, y, z \right) -f\left( x, y, z \right) }{h}
			\end{equation}
			only if $\lim_{h \to 0} \frac{g\left( h \right) }{h} = 0 \implies g\left( h \right)  = o\left( h \right) $
		\end{definition}
		\begin{example}
			For a function $f\left( x \right)  = x^2$, $f\left( x+h \right) -f(x) =  \left( x+h \right) ^2 - x^2 = 2xh + h^2$ 
		Note here that
		\begin{equation}
			\lim_{h \to 0} \frac{h^2}{h} = \lim_{h \to 0} h = 0
		\end{equation}
		since the function of h, $g\left( h \right)$ satisfies the previous definition, we say that $h^2$ is an $o\left( h \right) $, which further shows that $f'\left( x \right)  = 2x$
		\end{example}
		\begin{idea}
			With this, we can write out the derivative definition using just the numerator part of the derivative 
			\begin{equation}
				f\left( x+h \right)  - f\left( x \right) 
			\end{equation}
			And the result of this expression will leave us with two types of equations.
			\begin{equation}
				\text{derivative} \cdot  h
			\end{equation}
			or
			\begin{equation}
				\text{something} \cdot  o\left( h \right) 
			\end{equation}
			anything that is multipled by $o\left( h \right) $ will be reduced to zero due to the definition of $o\left( h \right) $, which is why the first expression gives us the derivative.\\
			This is especially useful when we are trying to find the derivative of a multivariable function. 
		\end{idea} 
	\item differentiability of a multivariable function
\begin{definition}
	we say $f$ is differentiable at $\vec{x}$ iff there exists $\vec{y}$ s.t. \[
	f(\vec{x}+\vec{h}) - f(\vec{x}) = \vec{y} \cdot \vec{h} + o(\vec{h})\\
	\vec{y} = Df(\vec{x}) = \text{the gradient of } f 
	.\] 
\end{definition}

\begin{example}
	Here we have the function $f\left( x, y \right)  = x+y^2$ and the h function $\vec{h} = \left( h_1,h_2 \right) $
	\begin{align*}
		f(\vec{x} + \vec{h}) - f(\vec{x}) &= f(x+h_1, y+h_2) - f(x, y)\\
						  &= x + h_1 + (y +h_2)^2 - x - y^2\\
						  &= h_1 + 2yh_2 + h_2^2\\
						  &= (1\hat{i} + 2y\hat{j})\cdot \vec{h} + h_2^2
	\end{align*}
	After this, we need to demonstrate that the remaining part of the function, that is $h_2^2$, is actually a $o\left( \vec{h} \right) $ function. To do that, we define $g(\vec{h}) = h^2 = (h_2\hat{j})\cdot (h_1\hat{i}+h_2\hat{j}) = h_2\hat{j}\cdot \vec{h}$. So we can write $g\left( \vec{h} \right)  = h_2\vec{j}\cdot \vec{h}$ 
\begin{align*}
	\frac{|g\left( \vec{h} \right) |}{\lVert \vec{h} \rVert } = \frac{|x| |h_2| \lVert \vec{h} \rVert |\cos \theta|}{\lVert \vec{h} \rVert } \le |xh_2|\\
	\lim_{h \to 0} |xh_2| = 0 \implies xh_2h_3 = o\left( \vec{h} \right) \\
	\therefore \nabla f\left( \vec{x} \right) = yz\hat{i} + xz\hat{i} + xy\hat{k}
.\end{align*}
	
	 We know that $h_2\to 0$ as $\vec{h}\to \vec{0}$ So $g(\vec{h})$ is $o(\vec{h})$ and we can claim that:
	 \begin{equation}
	 	\nabla f(\vec{x}) = 1\hat{i} + wy\hat{j}
	 \end{equation}
\end{example}
\begin{example}
	For this example we have $f\left( x, y, z \right)  = xyz$
	\begin{align*}
		f\left( \vec{x} + \vec{h} \right)  - f\left( \vec{x} \right)  &= \left( x+h_1 \right) \left( y+h_2 \right) \left( z+h_3 \right)  - xyz\\
		&= xyz + xyh_3 + xh_2z + xh_2h_3+h_1yz + h_1yh_3 + h_1h_2z + h_1h_2h_3 - xyz\\
		&=\left( yz\hat{i} + xz\hat{j} + xy\hat{k} \right) \cdot \vec{h}
	.\end{align*}
	We consider the case of $xh_2h_3 = g(\vec{h}) = xh_2\hat{k}\cdot \vec{h}$
	\begin{align*}
		\frac{g\left( \vec{h} \right) }{\lVert \vec{h} \rVert } = \frac{|x| |h_2| \lVert \vec{h} \rVert |\cos \theta| }{\lVert \vec{h} \rVert } \le |xh_2|\\
		\lim_{h \to 0} |xh_2| = 0 \implies xh_2h_3 \text{ is }  o\left( \vec{h} \right)\\
		\therefore \nabla f\left( \vec{x} \right)  = yz\hat{i} + xz\hat{j} + xy\hat{k}
	.\end{align*}
	\begin{equation}
		\nabla f\left( x,y,z \right)  = \left( f_x, f_y, f_z \right) 
	\end{equation}

\end{example}

\begin{theorem}
	For cartesian coordinates:
	\begin{equation}
		\nabla f(x, y, z) = \frac{\partial f}{\partial x}\hat{i} + \frac{\partial f}{\partial y} \hat{j}+ \frac{\partial f}{\partial z} \hat{k} 
	\end{equation}
	Or equally written as:
	\begin{equation}
		\nabla f(x, y, z) = (f_x, f_y, f_z)  
	\end{equation}
\end{theorem}
\item Gradients are the vectors that points steepest way "up the hill"
\item $\vec{x}$ is vector
\item $f(\vec{x}) $ is not vector
\item $\nabla f(\vec{x}) $ is a vector

\begin{example}
	Suppose we have a temperature function with respect to $x$ and $y$, and we have $\frac{\partial T}{\partial x} = 3 \frac{^\circ C}{m}$ and $\frac{\partial T}{\partial y} = 4 \frac{^\circ C}{m}$. From this, we can conclude that:
	\begin{equation}
		\nabla T = 3\hat{i}+4\hat{j}
	\end{equation}
	And that:
	\begin{equation}
		|\nabla T| = \sqrt{3^2+4^2}  = 5
	\end{equation} 
\end{example}
\begin{example}
	For this example we have $f\left( \vec{x} \right)  = xy^2z^3$, and we can compute their partial derivatives:
	\begin{align*}
		f_x &= y^2z^3 \\
		f_y &= 2xyz^3 \\
		f_z &=  3xy^2z^2 \\
	.\end{align*}
	And we have
	\begin{equation}
		\nabla f = \left( y^2z^3, 2xyz^3, 3xy^2z^2 \right) 
	\end{equation}
\end{example}
\begin{example}
	We have function $\vec{r}\left( x, y, z \right) \implies r = \sqrt{x^2+y^2+z^2}$
	So we then have
	\begin{align*}
		\nabla r &= \nabla \sqrt{x^2+y^2+z^2} \\ &= \frac{\frac{1}{2}2x}{\sqrt{x^2+y^2+z^2}} \hat{i} + \frac{\frac{1}{2}2y}{\sqrt{x^2+y^2+z^2}}\hat{j} + \frac{\frac{1}{2}2z}{\sqrt{x^2+y^2+z^2}}\hat{k}
	\end{align*}
	rewriting this gives us:
	\begin{equation}
		\nabla r = \frac{\vec{r}}{r}
	\end{equation}
\end{example}
\item For any directional derivative, we can define it being:
	\[
	\frac{\partial f}{\partial x} = \lim_{h \to 0} \frac{f\left( \vec{x_0} + h\hat{i} \right)  - f\left( \vec{x_0} \right) }{h}
	.\] 
	Expanding this to any arbituary direction is:
	\begin{definition}
		Directional derivative of function $f$ at $\vec{x_0}$ in direction $\hat{u}$ 
		\begin{equation}
			f_{\hat{u}} \left( \vec{x_0} \right)  = \lim_{h \to 0} \frac{f\left( \vec{x_0}+h\hat{u} \right)  - f\left( \vec{x_0} \right) }{h}
		\end{equation}
		We also have
		\begin{equation}
			f_{\hat{u}}\left( \vec{x_0} \right)  = \nabla f\left( \vec{x_0} \right) \cdot \hat{u}
		\end{equation}
	\end{definition}
\begin{proof}
	Proving that $f_{\hat{u}}\left( \vec{x_0} \right)  = \nabla f\left( \vec{x_0} \right) \cdot  \hat{u}$:
	\begin{equation}
		f\left( \vec{x} + \vec{h} \right)  - f\left( \vec{x} \right)  = \nabla f\left( \vec{x} \right) \cdot  \vec{h} + o\left( \vec{h} \right)  
	\end{equation}
	Where $\vec{h} = h\hat{i}$. Using this, we can show the above is equal to:
	\begin{equation}
		\nabla f\left( \vec{x} \right) \cdot h\hat{u} + o\left( \vec{h} \right) 
	\end{equation}
	and therefore:
	\begin{equation}
		\frac{f\left( \vec{x} + h\hat{u} \right)  - f\left( \vec{x} \right) }{h} = \nabla f \cdot  \hat{u} + \frac{o\left( \vec{h} \right) }{h}
	\end{equation}
	and taking the limit as $h\to 0$, we have the desired result.
\end{proof}

\begin{example}
	Using the same temperature example, we have $T\left( x, y \right) $ where $\frac{\partial T}{\partial y} = 4 \frac{^\circ C}{m}$ and $\frac{\partial T}{\partial x} = 3 \frac{^\circ C}{m}$. If we want to move in a direction of $\hat{u} = \cos\theta \hat{i} + \sin \theta \hat{j}$
	\begin{equation}
		T_{\hat{u}} = \left( \frac{\partial T}{\partial x} \hat{i} + \frac{\partial T}{\partial y} \hat{j} \right)\left( \cos \theta \hat{i} + \sin \theta \hat{j} \right)  = 3\cos\theta + 4\sin\theta
	\end{equation}
\end{example}
\begin{example}
	Suppose we have a parabolic hill described by $z\left( x,y \right)  = 20 - x^2 - y^{2}$ and we move straight up, or we can say that $\hat{u} = \left( 0,1 \right) $.
	\begin{align*}
		\frac{\partial f}{\partial x}  &= -2x \\
		\frac{\partial f}{\partial y}  &=  -2y \\
	.\end{align*}
	\begin{equation}
		\therefore z_{\hat{u}} = \left( -2x, -2y \right) \cdot \left( 0, -1 \right) = 2y
	\end{equation}
\end{example}
(The following is not on that lecture, but from Xue Qilin's notes)
\item Note that:
	\begin{align*}
		|f_{\hat{k}}\left( \vec{x} \right) | &= |\nabla f \cdot  \hat{u}|\\
		&= \lVert \nabla f \rVert \lVert \hat{u} \rVert |\cos\theta| \\
		&\le \lVert \nabla f \rVert	
	.\end{align*}
	\begin{example}
		Suppose that $z = f\left( x,y \right) = A + x + 2y + x^2 - 2y^2$ and we wish to find the steepest path down starting from $\left( 0,0,A \right) $. We know that:
		\begin{align*}
			\frac{\partial f}{\partial x}  &= 1-2x \\
			\frac{\partial f}{\partial y} &= 2-6y \\
		.\end{align*}
		such that:
		\begin{align*}
			\nabla f = \left( 1-2x \right) \hat{i} + \left( 2-6y \right) \hat{j} \implies - \nabla f = \left( 2x-1 \right) \hat{i} + \left( 6y-2 \right) \hat{j} 
		.\end{align*}
		The curve is given by:
		\begin{align*}
			\vec{r}\left( t \right)  = x\left( t \right) \hat{i} + y\left( t \right) \hat{j}
		.\end{align*}
		Where $x'\left( t \right)  = 2x\left( t \right) -1$ and $y'\left( t \right)  = 6y\left( t \right) -2$. This is in parametric form and we can convert to cartesian form by writing the derivatives as:
		\begin{equation}
			\frac{dy}{dx} = \frac{6y-2}{2x-1}
		\end{equation}
		and solving this differential equation to get:
		\begin{equation}
			3y = \left( 2x-1 \right) ^3 + 1
		\end{equation}
	\end{example}
\end{itemize}
