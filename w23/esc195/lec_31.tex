\section{Tangent Planes and Linear Approximations}
March 30, 2023
\begin{itemize}
	\item We were working with 2D curves, we are thinking these curves as LEVEL CURVES of 3D surfaces. We find that: Gradient vector is always normal to the tangent. We can use this property to find normal lines using gradient vector, and find tangent line using tangent vector.
		\[
		\vec{t} = \left( \frac{\partial f}{\partial y} ,-\frac{\partial f}{\partial x}  \right) 
		.\] 
		\begin{example}
			We have a circle $ c^2 + y^2 = 9$, or $f\left( x,y \right) = c$. We then have $f_x = 2x, f_y = 2y$\\
			Therefore, tangent line is:
			\begin{equation}
				\left( x-x_0 \right) 2x_0 + \left( y-y_0 \right) 2y_0 = 0
			\end{equation}
			USING
			\begin{equation}
				\left( x_0,y_0 \right)  = \left( \frac{3}{\sqrt{2}}, \frac{3}{\sqrt{2} } \right) 
			\end{equation}
			Tangent line:
			\begin{equation}
				\left( x-\frac{3}{\sqrt{2} } \right) 2 \frac{3}{\sqrt{2} } + \left( y-\frac{3}{\sqrt{2} } \right) 2 \frac{3}{\sqrt{2} } = 0 
			\end{equation}
			\begin{equation}
				y = \frac{6}{\sqrt{2} }-x
			\end{equation}
			This gives us a tangent line to the circle.\\
			At the same time, the normal line at that point is: $y=x$
		\end{example}
	\item Function of 3 variables:
		Level surfaces is $f\left( x,y,z \right) = C$. The level surface will be a surface in 3D plane:\\
		\begin{idea}
			Gradient vector $\nabla f$ is perpendicular to the surface, or we can say it's perpendicular to the tangent plane
		\end{idea}
\begin{example}
	we have $\vec{r}(t) = x(t)\hat{i} + y(t) \hat{j} + z(t) \hat{k}$ This is any curve on surface
	\begin{align*}
		\therefore f\left( x(t),y(t),z(t) \right)= f\left( \vec{r}(t) \right) = C \\
	\end{align*}
	we have: (Note here's some equation about chain rule, and how gradient dot tangent to this curve is 0\\
	From this we can conclude that the gradient will be perpendicular to any tangent line, showing it's normal to the whole surface\\
\end{example}
\begin{idea}
	tangent plane is formed by the tangent vector of any curve on the surface at that point
\end{idea}
\begin{example}
	Suppose we now have $f\left( x,y, z\right)=x^2+y^2+z^2 = 25 = C $ (this is a sphere)
	\begin{equation}
		\nabla f = \left( 2x,2y,2z \right) = 2\left( x,y,z \right) 
	\end{equation}
	Drawing a 3D curve showing the level surface, we have $\vec{x_0} = \left( x_0,y_0,z_0 \right) $ and we form a level plane. Any vector $\vec{x}$ on this level plane will:
	\begin{align*}
		\left( \vec{x}-\vec{x_0} \right) \text{ is perpendicular to }\nabla f\\
		\left( \vec{x} -\vec{x_0}\right) \cdot \nabla f\left( \vec{x_0} \right)  &= 0 \\
	.\end{align*}
	This is the equation of tangent plane to surface $f\left( x,y,z \right)  = C$ at $\vec{x_0}$
\end{example}
\begin{example}
	We have $x^2+y^2+z^2=25$, $\nabla f = 2\left( x,y,z \right) $. At point $\left( 0,5,0 \right) $ on this sphere, the tangent plane to that point will be a plane parallel to the xy plane, $y=5$\\
	We check this:
	 \begin{align*}
		 \left( x-x_0 \right) 2x_0 + \left( y-y_0 \right) 2y_0 + \left( z-z_0 \right) 2z_0 &=  0 \\
		 \left( y-5 \right) 10 &= 0 \\
		 y&= 5 \\
	.\end{align*}
	Checking the normal line, we expect a line that's the y axis:
	\begin{align*}
		\vec{r}(q) &= \vec{x_0} + q \nabla f\left( \vec{x_0} \right)  \\
		x &=  x_0 + qf_x \\
		y &= y_0 + qf_y \\
		z &=  z_0 + qf_z \\
		\text{so}\\
		x &=  0 \\
		y &=  5 + 10q \\
		z &=  0 \\
	.\end{align*}
\end{example}
\begin{example}
	Another example: $xy^2 + 2z^2 = 12$ at $\left( 1,2,2 \right) $ 
	\begin{align*}
		f_x &=  y^2 =  4 \\
		f_y &= 2xy =  4  \\
		f_z &=  4z =  8 \\
	.\end{align*}
	therefore the normal line is:
	\begin{align*}
		x &= 1+4q\\
		y &=  2+4q \\
		z &= 2 +8q\\
	.\end{align*}
\end{example}
\begin{warning}
Make sure the point actually lies on the plane!!!
\end{warning}
\begin{example}
	Having 2 elipsoid:
	\[
	f = x^2 +y^2 + z^2 - 8x-8y-6z+24 = 0\]
	\[
	g = x^2 + 3y^2 + 2z^2 = 9\]\[
	P\left( 2,1,1 \right) 
	.\] 
	Gradient of sphere:
	\begin{align*}
		\nabla f &=  \left( 2x-8 \right) \hat{i} + \left( 2y-8 \right) \hat{j} + \left( 2z-6 \right) \hat{k} \\
		\nabla f\left( 2,1,1 \right)  &= \left( -4,-6,-4 \right) \\
		\nabla g &=  2x\hat{i} + 6y\hat{j} + 4z\hat{k} \\
		\nabla g\left( 2,1,1 \right)  &= (4,6,4) = -\nabla f
	.\end{align*}
	This shows that $f$ and $g$ are parallel at that point, and that they touch at that point. So they SHARE a tangent plane
\end{example}
\begin{example}
	We now have a sphere and a parabaloid:
\[
f = x^2 + y^2 + z^2 - 4y-2z + 2 = 0\]\[
g = 3x^2 + 2y^2 -2z = 1\]\[
P\left( 1,1,2 \right) 
.\] 
\begin{align*}
	\nabla f &= 2x\hat{i} + \left( 2y-4 \right) \hat{j} + \left( 2z-2 \right) \hat{k} \\
	\nabla f\left( 1,1,2 \right)  &=  \left( 2,-2,2 \right) \\
	\nabla g &= 6x\hat{i} + 4y\hat{j} - 2\hat{k} \\
	\nabla g\left( 1,1,2 \right)  &=  \left( 6,4,-2 \right)  \\
.\end{align*}
Check their relationship, are they perpendicular?
\begin{equation}
	\nabla f\left( 1,1,2 \right)  \cdot  \nabla g\left( 1,1,2 \right)  = 12 - 8 - 4 = 0
\end{equation}
So these two surfaces are perpendicular at that point
\end{example}
\begin{example}
	A space curve:
	\[
	\vec{r}(t) = \frac{3}{2}\left( t^2+1 \right) \hat{i} + \left( t^4+1 \right) \hat{j} + t^3\hat{k}
	.\] 
	Ellipsoid:
	\[
	x^2 + 2y^2 + 3z^2 = 20
	.\] 
	Point:
	\[
	P(3,2,1)
	.\]
	So the line intersects the curve at that point, we want to show that the line is perpendicular to the curve at the intersection:\\
	For the surface:
	\begin{align*}
		\nabla f &= \left( 2x,4y,6z \right) \\
		\nabla f\left( 3,2,1 \right)  &=  \left( 6,8,6 \right)  \\
	.\end{align*}
	For the curve:
	\begin{align*}
		\vec{r}(t=1) &=  \left( 3,2,1 \right)  \\
		\vec{r}~' (t) &=  \left( 3t , 4t^3, 3t^2 \right)  \\
		\vec{r}~'\left( t=1 \right)  &=  \left( 3,4,3 \right)  = \frac{1}{2}\nabla f \\
	.\end{align*}
	So the direction vector of the curve is parallel to the normal vector of the surface, showing that this curve is normal to the plane at ths intersection
\end{example}
\begin{example}
	Ellipsoid:
	\[
	x^2 + y^2 + 2z^2 = 7
	.\] 
	On ths ellipsoid there's a bee at point $\left( 1,2,1 \right) $ and the bee is flying away normal to the surface, but hits a plane.\\
	Let's first find the normal line / direction vector:
	\begin{align*}
		\nabla f &=  \left( 2x,2y,4z \right)\\
		\implies \nabla f\left( 1,2,1 \right)  &= \left( 2,4,4 \right) 
	.\end{align*}
	Normal line will be:
	\begin{align*}
		x &=  1 + 2q \\
		y &=  2+4q \\
		z &=  1+4q \\
	.\end{align*}
	Position vector follows this line:
	\begin{equation}
		\vec{r}(q) = \left( 1+2q \right) \hat{i} + \left( 2+4q \right) \hat{j} + \left( 1+4q \right) \hat{k}
	\end{equation}
	we know the speed is 4 m/s, $\lVert \vec{r}~'(t) \rVert  = 4$ but $\lVert \vec{r} ~'(q) \rVert  = 6$. We can scale it down a bit, $q = \frac{2}{3}t$ so you have $\lVert \vec{r} ~ ' \left( \frac{2}{3}q \right) \rVert = 4$ 
	\begin{equation}
		\therefore \vec{r}(t) = \left( 1+\frac{4}{3}t \right) \hat{i} + \left( 2+ \frac{8}{5}t \right) \hat{j} + \left( 1+\frac{8}{3}t \right) \hat{k}
	\end{equation}
	Intersection of this line with the plane (when the bee hits the plane). The plane is $2x+3y+z = 49$. Sub in $x,y,z$ into the equation for the line:
	\begin{align*}
		2\left( 1+\frac{4}{3}t \right) + 3\left( 2+\frac{8}{5}t \right)  + \left( 1+\frac{8}{3}t \right) &= 49 \\
		\frac{40}{3}t &=  40 \\
		t &=  3 \\
	.\end{align*}
	So the bee hits the plane after 3s, and it ends up at: $\left( 5,10,9 \right) $
\end{example}
\end{itemize}
