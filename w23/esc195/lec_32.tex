
\begin{itemize}
	\item Last example for normal planes
		\begin{example}
			We have two surfaces: $S_1: xy = az^2$ and $S_2: z^2 = b-x^2-y^2$. Putting them in function form on level plane:
			\begin{align*}
				f\left( x,y,z \right)  = xy-az^2 &= 0\\
				g\left( x,y,z \right)  &=  x^2 + y^2 + z^2 = b \\
			.\end{align*}
			Then we have
			\begin{align*}
				\nabla f &= \left( y, x, -2az \right) \\
				\nabla g &=  \left( 2x,2y,2z \right)  \\	
			\end{align*}
			We have dot product of those two:
			\begin{equation}
				\nabla f \cdot  \nabla g = 4sy - 4az^2 = 4\left( xy-az^2 \right)  = 0 \\
			\end{equation}
			Note here, this tells us that the two surfaces are always normal to each other everywhere they intersect, since their tangent planes' normal vectors are normal.
		\end{example}

\end{itemize}
\section{Maximum and Minimum Values}
March 31, 2023
\begin{itemize}
	\item \textbf{14.7 Maximum and Minimum Values}
		\begin{definition}
			$f$ is said to have a local maximum at $\vec{x_0} \iff f\left( \vec{x_0} \right)  \ge f\left( \vec{x} \right) $ for $\vec{x}$ in some neighbourhood of $\vec{x_0}$ \\
			This is almost identical to the wording in single variable case, just replacing the original $x$ with $\vec{x}$ \\
			$f$ is said to have a local mimimum at $\vec{x_0} \iff f\left( \vec{x_0} \right) \le f\left( \vec{x} \right) $ for $\vec{x}$ in some neighbourhood of $\vec{x_0}$
		\end{definition}
		\begin{theorem}
			if $f$ has a local extreme value at $\vec{x_0}$, then either $\nabla f\left( \vec{x_0} \right)  = 0$ or $\nabla f\left( x_0 \right) $ DNE.\\
			This is just like single variable case.
		\end{theorem}
		\begin{proof}
			let $g\left( x \right) =f\left( x,y_0 \right) $ be function with a fixed $y_0$ value\\
			If $f$ has a maximum at $x_0$, then $g$ has a maximum at $x_0$. In this case, the derivative of $g$ will just be a partial derivative of $f$ with respect to $x$
			\begin{equation}
				\therefore \frac{dg}{dx}\left( x_0 \right)  = 0 = \frac{\partial f}{\partial x} \left( x_0,y_0 \right) 
			\end{equation}
		From this, lets suppose we have function $z = f\left( x,y \right)$ and make a function $h\left( x,y,z \right)  = z - f\left( x,y \right) $.\\ This implies $\nabla h = \left( -f_x, -f_y, 1 \right)  = \hat{k}$. This shows that the normal is perpendicular to the vector k.\\
		\end{proof}
		\begin{idea}
			if the planes have zero slope, than the normal is zero since $\nabla f = \vec{0}$
		\end{idea}
		\begin{definition}
			points where $\nabla f = \vec{0}$ or DNE are called critical points	
		\end{definition}
		\begin{definition}
			Points where $\nabla f = \vec{0}$ called stationary points (at that point in either direction you are not going up or down)
		\end{definition}
		\begin{definition}
			stationary points where are not local extremes are called saddle points
		\end{definition}
		\begin{idea}
			just because the tangent plane is horizontal, doesn't mean you have a maximum or a minimum
		\end{idea}
		\begin{example}
		Lets begin with a simple function $f\left( x,y, \right)  = 20 - x^2 - y^2$, $\nabla f = \left( -2x, -2y \right)$.\\
		We set $\nabla f = \vec{0} \implies -2x = 0, -2y = 0 \implies x = 0, y = 0$ 
		\begin{equation}
			\therefore (0,0) \text{ is a stationary point.}
		\end{equation}
		set $x = h, y = k$ where $h,k$ are some tiny numbers. At origin, $f\left( 0,0 \right)  = 20, f\left( h,k \right) = 20-h^2-k^2 < 20~ \forall h,k>0$
		\end{example}
		\begin{example}
			another surface $f\left( x,y \right)  = xy$, $\nabla f = \left( y,x \right) $ \\
			Set $\nabla f = \left( 0,0 \right)  \implies y = 0, x = 0$ \\
			$f\left( 0,0 \right)  = 0$, $f\left( h,k \right)  = hk$ \\
			For $h,k > 0$ or $h,k < 0$, $hk>0$ \\
			But if we have them alternating signs, $hk < 0$, showing that this is a saddle point
		\end{example}
		\begin{example}
			$f\left( x,y \right)  = 2x^2 + y^2 - xy - 7y\implies\nabla f = \left( 4x-y, 2y-x-7 \right) $\\
			$\nabla f = \vec{0} \implies y = 4x, yx-x-7 = 0 \implies x = 1, y = 4$\\
			$f\left( 1,4 \right)  = 2 + 16 - 4 - 28 = 14$
			\begin{align*}
				f\left( 1.01, 4.01 \right)  &= -13.9998\\
				f\left( 1,01,3.99 \right)  &=  -13.9996 \\
				f\left( 0.99, 4.01 \right) &= -13.9996\\
				f\left( 0.99, 3.99 \right) &= -13.9998 \\
			\end{align*}
			This seems to be a minimum point
		\end{example}
	\item It's perfectly possible for a maximum or a minimum that occurs along a line!\\
		Using a donut shape as an example, the equation is \[
		f\left( x,y \right)  = z =  \left( a^2 - \left( \sqrt{x^2 + y^2} -R \right) ^2 \right) ^\frac{1}{2}
		.\] 
		\[
			\nabla f = \left[ \frac{1}{2}\left( \ldots \right) ^{-\frac{1}{2}}\left( -2 \right) \left( \sqrt{\ldots} -R \right) \frac{1}{2}\left( x^2 + y^2 \right) ^{-\frac{1}{2}} \cdot 2x, \frac{1}{2}\left( \ldots \right) ^{-\frac{1}{2}}\left( -2 \right) \left( \sqrt{\ldots} -R \right) \frac{1}{2}\left( x^2 + y^2 \right) ^{-\frac{1}{2}} \cdot  2y\right] 
		.\] 
		\[
		 \nabla f = 0 \implies \sqrt{x^2 + y^2 = R}\\
		 f\left( x^2 + y^2 = R^2 \right)  = \left( a^2 - \left( \sqrt{R^2}  - R \right) ^2 \right) ^{\frac{1}{2}} = a\\
		.\] 
		This is the top surface of the donut shape, and shows that the critical point CAN be a curve.\\
		(question): what about neighboring point, aren't they the "same" height?
		\begin{example}
			cone $f\left( x,y \right)  = -\sqrt{x^2 + y^2} $, $\nabla f = \left( -\left( x^2 + y^2 \right) ^{-\frac{1}{2}}\cdot 2x, -\left( x^2 + y^2 \right) ^{-\frac{1}{2}} \cdot 2y \right) $ This is DNE at $\left( 0,0 \right) $
		\end{example}
		\begin{theorem}
			Second Derivative\textbf{s} Test:\\
			Note the strategic "s" at the end of the "derivatives"?\\
			For $f\left( x,y \right) $ with continuous second order partial, and $\nabla f\left( x_0,y_0 \right)  = 0$, set \[
			A = \frac{\partial ^2f\left( x_0,y_0 \right) }{\partial x^2} , B = \frac{\partial ^2f\left( x_0,y_0 \right)}{\partial x \partial y}, C = \frac{\partial ^2f\left( x_0,y_0 \right) }{\partial y^2} 
			.\] 
			and form the discriminant $ D = AC-B^2$\\
			\begin{enumerate}
				\item if $D < 0$, then $\left( x_0,y_0 \right) $ is a saddle point\\
				\item if $D > 0$, $A,C > 0$ $\left( x_0,y_0 \right) $ is a local minimum\\
				\item if $D > 0$, $A,C < 0$ $\left( x_0,y_0 \right) $ is a local maximum\\
				\item if $ D = 0$, you know nothing
			\end{enumerate}
			(question) can $D=0$?
		\end{theorem}
		\begin{example}
			$f\left( x,y \right) = xy$ 
			\[
			f_x = y, f_{xy} = 0 = A\\
			f_y = x, f_{yy} = 0 = C\\
			f_{xy} = 1 = B\\
			.\] 
			$\therefore D = AC-B^2 = -1 < 0$, therefore $\left( 0,0 \right) $ is a saddle point.
		\end{example}
		\begin{example}
			$f\left( x,y,z \right)  = 2x^2 + y^2 -xy - 7y, \nabla f = 0$ at $\left( 1,4 \right) , f\left( 1,4 \right)  = -14$ 
			\begin{align*}
				f_x &= 4x-y\\
				f_y &= 2y-x-7 \\
				f_{x x} &= 4 = A\\
				f_{yy} &= 2 = C\\
				f_{xy} &= -1 = B \\
			.\end{align*}
			$ \therefore  D = 8 - 1 = 7 > 0$ and $A = 4 > 0$ which shows it's a local minimum
		\end{example}
	\item Absolute Extreme Values\\
		\begin{theorem}
			if $f$ is continuous on a bounded, closed set, then $f$ takes on both an absolute minimum and an absolute maximum on that set.\\
			Basically: if domain is not infinite, we have both abs max and abs min
		\end{theorem}
		\begin{example}
			$f\left( x,y \right)  = \left( x-4 \right) ^2 + y^2$, we want to find abs max and min\\
			On set $\left\{ \left( x,y \right) : 0 \le  x \le  2, x^3 \le  y \le 4x \right\} $. Geometrically, this function's input is the area between the $ y = 4x$ and $y = x^3$ curves for $x \ge  0$\\
			We first start looking for INTERNAL extreme values:
			\[
			\nabla f = \left( 2\left( x-4 \right) , -2y \right) \]\[
			 \nabla f = 0 \implies x = 4, y = 0. \text{ This is not within the set. }
			.\] 
		Thus in this example, what we really care about is the endpoints.\\
		However, note that in this case, the abs max and min can occur on the entire boundary.\\
		\\
		Part 1: boundary $y = x^3, 0 \le  x\le 2$\\
		We first parametrize the curve:
		\begin{align*}
			x = t, y = t^3\\
			\vec{r_1} &=  t\hat{i} + t^3\hat{j}, 0 \le  t \le  2 \\
			f_1(t) &= f\left( \vec{r_1}\left( t \right)  \right) ~ \text{What is the max/min along this?}\\
			f'(t) = \nabla f \cdot  \vec{r} ~ ' &= \left( 2\left( t-4 \right) , 2t^3 \right) \cdot \left( 1, 3t^2 \right) \\
												&= 2t-8t+6t^5\\
		.\end{align*}
		Set $f_1'(t) = 0 \implies t\left( 1+3t^4 \right)  = 4 \implies t = 1$ is the only solution. This gives us the point $(1,1)$ which is indeed on our curve. So this is a useful point to know things about. Note $f(1,1) = 10$\\
		At that point,  $f_1''(t) = 2 + 30t^4 = 32 > 0$ at $(1,1)$, so local min.\\
		\\
		Part 2: $ y = 4x$ boundary: $x = t, y = 4t, 0 \le  t \le  2$
		\begin{align*}
			f_2(t) &= \left( t-4 \right) ^2 + \left( 4t \right) ^2 = 17t^2 - 8t + 16\\
			f_2'(t) &=  34t-8 \\
		.\end{align*}
		Set $f_2'(t) = 0\implies t = \frac{8}{34} = \frac{4}{17}\implies \left( \frac{4}{17}, \frac{16}{17} \right) $ \\
		Eval 2nd derivative: $f_2''(t) = 34 > 0$, local minimum at $f\left( \frac{4}{17}, \frac{16}{17}  \right) = 15.06$ 
		\\
		\\
		We are missing MAXIMUM!!! For this, we have to check the endpoints, where the two curves intersect:\\
		$f(0,0) = 16, f(2,8) = 68 \implies f(1,1) \text{ is abs min}, f(2,8) \text{ is abs max}$
		\end{example}
		\begin{idea}
			Four things to check for abs max/min:\\
			\begin{enumerate}
				\item Check where $\nabla f$ DNE\\
		\item Check where $\nabla f= \vec{0}$\\
		\item Check boundaries\\
		\item check end points of boundaries
			\end{enumerate}
		\end{idea}
		\begin{example}
			$f\left( x,y \right)  = xy^2 - x$ on $\left\{ \left( x,y \right) | x^2 + y^2 \le  3 \right\} $ \\
			Find partials:
			\begin{align*}
				f_x = y^2-1, f_{xx} = 0\\
				f_y = 2xy, f_{yy} = 2x\\
				f_{xy} = 2y
			.\end{align*}
			We then have:
			\begin{align*}
				\nabla f = \vec{0} \implies y^2 - 1 = 0, 2xy = 0\\
				y &= \pm 1 \\
				x&= 0 \\
				\text{zeros at: } \left( 0,1 \right) , \left(  0,-1\right) 			
			.\end{align*}
			$\left( 0,1 \right): A= f_{xx} = 0, C = f_{yy} = 2x = 0, B = f_{xy} = 2y = 2$ Which:
			\[
			D = AC-B^2 = -4 < 0,~ \therefore f(0,1)\text{ is a saddle pt.}
			.\] 
			$\left( 0,-1 \right): A= f_{xx} = 0, C = f_{yy} = 2x = 0, B = f_{xy} = 2y = -2$ Which:
			\[
			D = AC-B^2 = -4 < 0,~ \therefore f(0,-1)\text{ is a saddle pt.}
			.\] 
			Now lets check boundary:
			$x^2 + y^2 = 3 \implies y^2 = 3 - x^2$, we have $f\left( y^2 = 3-x^2 \right)  = x\left( 3-x^2 \right) -x = 2x-x^3 = f_1(x)$ 
			Find max along this curve using traditional methods:
			\[
			f_1'(x) = 2-3x^2, f_1'(x) = 0 \implies x^2 = \frac{2}{3} \implies x = \pm \sqrt{\frac{2}{3}}, y = \pm \sqrt{\frac{7}{3}} 
			.\] 
			Second derivative test:
			\[
			f_1''(x) = -6x \implies f_1''\left(\sqrt{\frac{2}{3}}\right) = -6\sqrt{\frac{2}{3}}  < 0 \text{ Local max}
			.\] 
			\[
			f_1''\left( -\sqrt{\frac{2}{3}}  \right)  = 6\sqrt{\frac{2}{3}} > 0 \text{ local min} 
			.\] 
We have two internal saddle points, and we then have 2 maximum values, and two minimum values along the curve.\\
\textbf{we are missing something}: If it's a single variable question, we can't have 2 maximum and 2 minimum (as we did this along the outer circle edge, implying something is missing. \\
We are missing the endpoints:
\begin{align*}
	x = -\sqrt{3},  y &= 0 \implies f = \sqrt{3}   \\
	x = \sqrt{3} y &=  0 \implies f = -\sqrt{3}  \\
\end{align*}
Then we have $f\left( -\sqrt{3} ,0 \right)  = \sqrt{3} $ is abs max, $f\left( \sqrt{3} , 0 \right) = -\sqrt{3} $ is abs min.\\
This is why we don't usually convert it to 1 single variable, if we parametrize it: 
\[
\vec{r}(t) = \sqrt{3} \cos t \hat{i} + \sqrt{ 3}  \sin t \hat{j}
.\] Using this, we get all 6 points
		\end{example}

\end{itemize}
