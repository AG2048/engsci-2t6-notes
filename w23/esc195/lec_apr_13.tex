\section{14.10 Differentiability of an integral with respect to its parameter}
April 13, 2023
\begin{itemize}
	\item $F\left( x \right)  = \int_c^d f\left( x,y \right) dy$ 
		\begin{theorem}
			If, in the closed rectangle $x \in [a,b]$ and  $y \in [c,d]$, the function  $f\left( x,y \right) $ has a continuous derivative with respect to x, then for $x\in[a,b]$:
			 \begin{align*}
				\frac{dF}{dx} = \frac{d}{dx} \int_c^d f\left( x,y \right) dy = \int_c^d \frac{\partial f}{\partial x} dy
			.\end{align*}
		\end{theorem}
		\begin{proof}
			Given $x$ and $x+h \in [a,b]$, then:
			\begin{align*}
				\frac{F(x+h) - F(x)}{h} &= \frac{1}{h}\int_c^df\left( x+h, y \right) dy - \frac{1}{h}\int_{c}^{d} f\left( x,y \right) dy\\
&= \int_{c}^{d} \frac{f\left( x+h,y \right) -f\left( x,y \right) }{h}dy  \\
			.\end{align*}
			\begin{align*}
				\frac{dF}{dx} = \lim_{h \to 0} \int_{c}^{d} \frac{f\left( x+h,y \right) -f\left( x,y \right) }{h}dy = \int_{c}^{d} \frac{\partial f}{\partial x} dy  
			.\end{align*}
		\end{proof}
		\begin{example}
			$F\left( x \right)  = \int_{2}^{4} e^{xy}dy $ 
			\begin{align*}
				\frac{dF}{dx} &= \frac{d}{dx}\left[ \frac{e^{xy}}{x} \right]_2^4 \\
				&= \frac{d}{dx}\left[ \frac{e^{4x}-e^{2x}2}{2} \right]  \\
				&= e^{4x}\left( \frac{4x-1}{x^2} \right) -e^{2x}\left( \frac{2x-1}{x^2} \right)  \\
			.\end{align*}
			\begin{align*}
				\frac{dF}{dx} &= \int_{2}^{4} \frac{\partial e^{xy}}{\partial x} dy \\
				&= \int_{2}^{4} ye^{xy}dy  \\
				&= \left[ \frac{y}{x} e^{xy} \right]_2^4 -\int_{2}^{4} \frac{e^{xy}}{x}dy \\
				&= \left[ \left( \frac{y}{x}-\frac{1}{x^2} \right) e^{xy} \right] _2^4 \\
				&= e^{4x}\left( \frac{4}{x}-\frac{1}{x^2} \right) -e^{2x}\left( \frac{2}{x}-\frac{1}{x^2} \right)  \\
			.\end{align*}
			We get the same result
		\end{example}
		\begin{example}
			we have $A(t) = \int_{x_1(t)}^{x_2(t)} f(x)dx, f(x)\ge 0 $ $ A $ is the area under the graph with a "sliding" window made by $x_1(t), x_2(t)$\\
			Basically, every time we want to calculate $\Delta A$, we have to add the new area, subtract the old area "lost".
			\begin{align*}
				\Delta A &= A\left( t+\Delta t \right) - A\left( t \right) \\
						 &= \int_{x_2(t)}^{x_{2}\left( t+\Delta t \right) } f\left( x \right) dx  - \int_{x_1(t)}^{x_1(t+\Delta t)} f(x)dx \\
			.\end{align*}
			\begin{align*}
				\int_{a}^{b} f\left( x \right) dx &= f\left( x^* \right) (b-a) && x^* \in (a,b)\\
				\int_{z}^{z+\Delta z} f(x)dx &= f\left( z^* \right) \Delta z  && z^* \in \left( z, z+\Delta z \right) \\ 
			.\end{align*}
			\begin{align*}
			\Delta A = f\left( x_2^* \right) \Delta x_2 - f\left( x_1^* \right) \Delta x_1
			\end{align*}
			Where $x_2^* \in \left( x_2, x_2+\Delta x_2 \right) , x_1^* \in \left( x_1, x_1+\Delta x_1 \right) $ 
			\begin{align*}
				\therefore \frac{\Delta A}{\Delta t} &= f\left( x_2^* \right) \frac{\Delta x_2}{\Delta t} - f\left( x_1^* \right) \frac{\Delta x_1}{\Delta t}  \\
			\implies \frac{dA(t)}{dt} &=  f\left( x_2 \right) \frac{dx_2}{dt}-f\left( x_1 \right) \frac{dx_1}{dt} 
			\end{align*}
		\end{example}
		\begin{theorem}
			Leibnitz's Rule: Given a region $R$ in the $x-y$ plane in which the functions $\phi_1(x)$ and $\phi_2(x)$ have continuous derivatives with respect to x, and in which $f\left( x,y \right)$ is continuously differentiable:\\
			if $F(x) = \int_{y = \phi_1(x)}^{y = \phi_2(x)} f\left( x,y \right) dy $ \\
			Then 
			$$
			\frac{dF}{dx} = \int_{\phi_1(x)}^{\phi_2(x)} \frac{\partial f}{\partial x} dy + f\left( x,y=\phi_2(x) \right) \frac{d\phi_2}{dx} - f\left( x,y =\phi_1(x)\right)  \frac{d\phi_1}{dx}
			$$
		\end{theorem}
		\begin{example}
			$f\left( x \right)  = \int_{0}^{x^2} \sin\left( xy \right) dy $ 
			This gives:
			\begin{align*}
				f\left( x,y \right)  &= \sin(xy) && \frac{\partial f}{\partial x}  = y\cos(xy)\\
				\phi_1(x) &= 0 && \frac{d\phi_1}{dx} = 0 \\
				\phi_2(x) &=  x^2 && \frac{d\phi_2}{dx} = 2x \\
			.\end{align*}
			$$
\therefore \frac{dF}{dx} = \left(\int_{0}^{x^2} y \cos (xy) dy\right)  + 2x \sin(x^3) - 0 \sin 0
			$$
		\end{example}
		\begin{example}
			$F\left( x \right) = \int_{0}^{1} \frac{y^{x}-1}{\ln y}dy, x > -1 $:
			\begin{align*}
				F'(x) &=  \frac{d}{dx} \int_{0}^{1} \frac{y^{x}-1}{\ln y}dy  \\
				&= \int_{0}^{1} \frac{\partial }{\partial x} \left( \frac{y^{x}-1}{\ln y} \right) dy  \\
				&= \int_{0}^{1} \frac{y^{x}\ln y}{\ln y}dy  \\
				&= \int_{0}^{1} y^{x}dx  \\
				&= \left[ \frac{y^{x + 1}}{x + 1} \right] _0 ^ 1 \\
				&= \frac{1}{x+1} \\
			.\end{align*}
			\[
			\therefore F(x) = \int \frac{dx}{x+1} = \ln \left| x + 1 \right| + C
			.\] 
			\[
			 \implies F(0) = \int_{0}^{1} \frac{y^0 - 1}{\ln y} dy = \int_{0}^{1} \frac{1-1}{\ln y}dy = 0  
			.\] 
			\[
			\therefore C = 0, \therefore  F(x) = \ln(x+1), x>-1
			.\] 
		\end{example}
\end{itemize}
\section{Exam prep}
\begin{example}
	Question 1: Basic Integral (basically free questions)
\end{example}
\begin{example}
	Question 2: It's actually the best question:\\
	Easiest polar coordinate question, what's the area between curves, with easy to find intersections\ldots
\end{example}
\begin{example}
	Question 3(a): Sequence:
	\[
		a_n = \mathrm{max}\,\left\{ \sin 1, \sin 2, \sin 3, \ldots, \sin n \right\} 
	.\] 
	Does this have a limit?\\
	Idea: monotonic sequence theorem, non-decreasing, with lowst upper bound, therefore sequence converges to its upper bound, which is 1\\
	Does it ever reach it? no, because $\pi$ is irrational. But it should become infinitely close.
\end{example}
\begin{example}
	Question 3(b): Series: Does it converge?
	\[
	\sum_{n=2}^{\infty} \frac{5^n}{n}x^n
	.\] 
	Ratio test or root test would work. Root test is marginally less work than ratio test.
	\[
	a_n^{\frac{1}{n}} = \frac{5 \left| x \right| }{n^{\frac{1}{n}}} \to 5\left| x \right| 
	.\] 
	Then remember to test the endpoints of $\frac{1}{5}$ or $-\frac{1}{5}$
\end{example}
\begin{example}
	Question 4: a proof test:
	\[
	\lim_{n \to \infty} \frac{a_n}{b_n} = 0
	.\] 
	tells us 
	\[
	\left| \frac{a_n}{b_n} \right| < 1\to a_n < b_n
	.\] 
	given that $b_n$ converges, $a_n$ converges. (This part is already 5 marks)\
	If approach infinity it's the other way around\\
	if approach a number, $a_n$ is between a range of  $b_n$ and vice versa. If any converge we can argue other one converge.
\end{example}
\begin{example}
	Question 5: part c can be done before part a and b\\
	We can do integral test:\\
	One way the riemann sum rectangle is below the curve, one way it's above the curve, so it's the inequality lawl (left hand riemann sum and right hand riemann sum)\\
	part b:\\
	Sum of log is just product of the contents.
	\[
	\sum_{k=1}^{n} \ln k = \ln (n!)
	.\] 
\end{example}

\begin{example}
	question 6a:\\
	We use taylor series:
	\[
	f\left( x \right)  = f(a) + f'(a) (x-a) + \frac{f''(a)}{2}(x-a)^2\ldots
	.\] 
	g(x) is the same\\
	 \[
	\frac{f(x)}{g(x)}=\frac{f'(a)(x-a) + \frac{1}{2}f''(a)(x-a)^2\ldots}{g'(a)(x-a) + \frac{1}{2}g''(x) (x-a)^2\ldots}
	.\] 
	cancel out the (x-a) we get:
	\[
	\frac{f'(a) + \frac{1}{2}f''(a) (x-a)\ldots}{g'(a) + \frac{1}{2}g''(a) (x-a)}
	.\] 
	here limit is as $x\to a$ so we have l'hopital's rule
\end{example}
\begin{example}
	question 6b:
	\[
	1 + \frac{x}{2} + \frac{x^2}{4!} + \frac{x^3}{6!} + \frac{x^4}{8!} + \ldots = 0
	.\] 
	This is taylor series of some function that looks like cosine?
	\[
	f(-t^2) = 1 - \frac{t^2}{2!} + \frac{t^4}{4!} - \frac{t^6}{6!} + \frac{t^8}{8!} + \ldots = \cos(t)
	.\] 
	\[
	\implies \cos t = 0 \implies t = \pm \left( \frac{\pi}{2}-\pi k \right) 
	.\] 
	\[
	f(x) = 0 \implies x = t^2 \implies x = -t^2 \implies x = -\left( \frac{\pi}{2} - \pi k \right) ^2
	.\] 
\end{example}
\begin{example}
	Question 7: Advice: Be careful and methodical for this type of question! Any negative signs lost is huge loss\\
	\[
	\vec{r}(t) = t\hat{i} + \frac{1}{t}\hat{j} + \sqrt{2} \ln(t) \hat{k}
	.\] 
	find unit tangent, unit normal, tangential and normal of acceleration at t = 1
\end{example}
\begin{example}
	Question 8(a) does this limit exist?
	\[
	\lim_{\left( x,y \right)  \to (0,0)} \frac{2xy}{x^2 + 3y^2} 
	.\] 
	No cuz set x = y and set x = 0..
\end{example}
\begin{example}
	8(b)\\
	Formal definition of multivariable function $o(h)$ to find gradient of $f\left( x,y \right)  = \frac{1}{2}x^2 + 2xy + y^2$\\
	\[
	f\left( x+h_1, y + h_2 \right)  - f\left( x,y \right)  = \nabla f \cdot  \vec{h} + o(h), \vec{h} = (h_1,h_2)
	.\] 
\end{example}
\begin{example}
	Question 9:\\
	Abs max and min of $f\left( x,y \right) = x^2 + 2y^2 - 2x - 4y + 1$ on
	\[
	D = \left\{ (x,y) | 0 \le x \le  2, 0 \le  y \le  3 \right\} 
	.\] 
	Sketch region and show locations of critical points\\
	This is not really a lagrange multiplier question, don't do it\\
	To solve:
	\[
	\nabla f = \left( 2x - 2, 4y - 4 \right)  = \vec{0}
	.\] 
	Then test the 4 lines  $ x = 0, x = 2, y = 0, y = 3$ and 4 points  $(0,0), (2,3), (2,0), (0,3)$\\
	To test lines: In this scenario just treat it as a 1d function by subbing the numbers in
\end{example}
\begin{example}
	10, lagrange multiplier question
	\[
	f\left( x,y,z \right)  = x^2 + y^2 + 2z, \text{ on } h\left( x,y,z \right)  = x + y + 2z = 2, g\left( x,y,z  \right) = x^2 + y^2 - z = 0
	.\] 
	\begin{align*}
		\nabla f &=  \left( 2x,2y,2 \right)  \\
		\nabla h &= (1,1,2)\\
		\nabla g &= \left( 2x,2y,-1 \right)  \\
	.\end{align*}
	\begin{align*}
		2x &= \lambda 1 + \mu \cdot  2x \\
		2y &= \lambda 1 + \mu \cdot  2y \\
		2 &= \lambda * 2 - \mu \cdot  1 \\
	.\end{align*}
	Solve for it
\end{example}
\begin{example}
	Question 11\\
	\[
	f\left( x,y,z \right)  = e^{x}\ln z + x^2y - z \cos y + C
	.\] 
	Find if the 2nd derivatives match:
	\[
	f_x = e^{x}\ln z + 2xy, f_y = x^2 + z \sin y, f_z = \frac{e^x}{z} - \cos y
	.\] 
	\[
	f_{xy} = f_{yx} = 2x
	.\] 
	\[
	f_{xz} = f_{zx} = \frac{e^x}{z}
	.\] 
	\[
	f_{yz} = f_{zy} = \sin y
	.\] 
\end{example}

\begin{example}
	$ y (x) = 2 + \int_{1}^{x} \left( y(t) \right) ^2dt $ 
	\[
	y'(x) = \left( y(x) \right) ^2\implies \frac{dy}{y^2} = dx\implies y  = \frac{-1}{x+C}
	.\] 
	But note y is defined as definite integral, no constant term. Try $y(1) = 2 = \frac{-1}{1+C} \implies y(x) = \frac{2}{3-2x}$
\end{example}
\begin{example}
	Turn this into a workable partial fraction:
	\[
		\int_{0}^{\frac{\pi}{4}} \sqrt{\tan x} dx
	.\] 
	We try $u = \sqrt{\tan x}, 0 \le  u \le  1 $\\
	$ 2udu = \sec ^2 x dx = \left(  1 + \tan ^2 x\right) dx = \left( 1 + u^{4} \right) du$
	So the integral is:
	\[
	2 \int_{0}^{1} \frac{u^2}{1 + u^{4}}du 
	.\] 
	Partial Fractions: 
	\[
	\frac{u^2}{1 + u^{4}} = \frac{Au + B}{u^2 + \sqrt{2} u + 1} + \frac{Cu + D}{u^2 - \sqrt{2} u + 1}
	.\] 

\end{example}
